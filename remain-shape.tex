\subsection{Remaining Work: An Exploratory Study to Discover the Differences and Similarities between the Pattern-based Approach and the Concept-based Approach}
\label{sec:remaining-shape}

As the last part of the proposed future work, I plan to perform an empirical comparison between the pattern-based approach and the concept-based approach to discover the differences and similarities between them.
%
For the first step of this study, a further evaluation of the concept-based approach will be conducted as a developer-oriented survey with a selected group of experienced developers.
%
In the planned survey questions, we will ask them about how they write their unit tests and how they modify the existing tests.
%
Depending on the results from this exploratory study, we could not only gain more insight on how developers name their tests but also use the analyzed result of the survey to set a set of comparison metrics when comparing the pattern-based approach and the concept-based approach (i.e., like which approach can suggest test names that is more likely to meet the approval of developers).
%
The exploratory study will conducted as the several steps mentioned in~\cref{sec:introduction}.
%
When the comparison of the two approaches is completed, we will perform a quantitative analysis of the result and summarize our findings.
%
If there is a potential overlap between the two approaches (i.e., some uniqueness-based test names might be the same names that the pattern-based approach would suggest), we will discuss the possibility to construct a framework using the combination of the two approaches.
%
This framework can be utilized to construct test names that are both descriptive about its test body but also unique in its associated test class, and the test names that the framework produces could be a potentially excellent match that can meet developer's approval.

\subsection{Research Plan}

The research plan of all proposed work contains four steps.
%
The first step has been done in my preliminary work that is to build a pattern-based approach to detect and improve non-descriptive test names.
%
The second step is to conduct an empirical study about if tests are named after what makes them unique and gain the knowledge of basic naming rationales, which is the foundation of all following work.
%
The third step is to develop and implement a concept-based approach to tag the tests with a specific block of text that shows the uniqueness of tests, and improve existing test names with the extracted uniqueness.
%
This step will be done as the first part of the planned future work.
%
The last step is targeting at investigating the differences and similarities between the patter-based approach and the concept-based approach with an exploratory study.