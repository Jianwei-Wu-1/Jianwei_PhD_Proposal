\subsection{Remaining Work: An Fully Automated Framework to Provide Descriptive Test Names for Developers}
\label{sec:remaining-work2}

As the last part of the proposed future work, I plan to propose an automated framework that combined the pattern-based approach and the uniqueness-based approach.
%
For the first step of the exploratory study mentioned in~\cref{sec:introduction}, a further evaluation of the pattern-based and the uniqueness-based approaches will be conducted as a developer-oriented survey with a selected group of experienced developers.
%
In the planned survey questions, we will ask them about how they write their unit tests and how they modify the existing tests.
%
Analyzing from the results of the developer-oriented survey, we could both have a better understanding of how developers name their tests and gain a general sense about when to apply either approach.
%
If it is possible to combine the two approaches, we will explore the possibility to construct the automated framework using the combination of the two approaches.
%
The framework can be utilized to detect non-descriptive test names and generate uniqueness-based descriptive test names for them.
%
Moreover, the framework is designed to have a fully automated process, so there is no developer assistance needed for detecting non-descriptive test names and replacing them with descriptive names that are constructed by what makes those tests unique.


\subsection{Research Plan}

The research plan of all proposed work contains four steps.
%
The first step has been done in my preliminary work that is to build a pattern-based approach to detect and improve non-descriptive test names.
%
The second step is conducting an empirical study to investigate if tests are named after what makes them unique and using the collected result of the study to motivate a uniqueness-based naming rationale for descriptive test name generation.
%
The following steps will be our planned future work.
%
The third step is to develop and evaluate the uniqueness-based approach that can provide uniqueness-based descriptive test names for existing JUnit tests.
%
The last step is targeting at exploring a possible combination of the patter-based approach and the uniqueness-based approach to construct a fully automated framework that can detect tests without descriptive names and generate descriptive names for them.