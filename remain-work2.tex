\subsection{Remaining Work: An Fully Automated Framework to Provide Descriptive Test Names for Developers}
\label{sec:remaining-work2}

As the last part of the proposed future work, I plan to propose an automated framework that is the combination of the pattern-based and the uniqueness-based approach.
%
For the first step of the exploratory study mentioned in~\cref{sec:introduction}, a further evaluation of the pattern-based and the uniqueness-based approaches will be conducted as a developer-oriented survey with a selected group of experienced developers.
%
In the planned survey questions, we will ask them about how they write their unit tests and how they modify the existing tests.
%
Analyzing from the results of the developer-oriented survey, we could not only have a better understanding of how developers name their tests but also gain a general sense about when to apply the pattern-based approach to detect non-descriptive test names among existing tests and apply the uniqueness-based approach to generate descriptive names for them.
%
If it is possible to combine the two approaches (i.e., when to apply each approach), we will explore the possibility to construct the automated framework using the combination of the two approaches.
%
The framework can be utilized to construct test names that are both descriptive about its test body but also unique in its associated test class.
%
Moreover, the framework is designed to have a fully automated process, so there is no developer assistance needed for detecting non-descriptive test names and replacing them with descriptive names that are constructed by what makes those tests unique.
https://www.overleaf.com/project/5dc87600da658500013ef99c
\subsection{Research Plan}

The research plan of all proposed work contains four steps.
%
The first step has been done in my preliminary work that is to build a pattern-based approach to detect and improve non-descriptive test names.
%
The second step is conducting an empirical study to investigate if tests are named after what makes them unique and using the collected result of the study to motivate a uniqueness-based naming rationale for descriptive test name generation.
%
The following steps will be our planned future work.
%
The third step is to develop and implement a uniqueness-based approach that is based on the uniqueness-based naming rationale to tag each test with the information that makes the test unique, and we can either improve existing test names with the extracted uniqueness or generate uniqueness-based test names.
%
The last step is targeting at exploring a possible combination of the patter-based approach and the uniqueness-based approach to construct a fully automated framework that can detect non-descriptive test names and replace them with descriptive ones.