\section{Foreseen Contributions}
\label{sec:contributions}

In this proposal, I propose a systematic approach to provide descriptive names for unit tests.
%
In the preliminary work, first, I developed a pattern-based approach to detect and improve non-descriptive test names.
%
Second, an empirical study to investigate if unit tests are named after what makes them unique was conducted to motivate the following work.
%
Based on the findings of the empirical study, I further proposed an automated approach to identify unique attributes of unit tests.
%
Finally, I plan to create a name-generation approach to provide descriptive test names that meet with develop approval.
%
Moreover, this approach would be built on top of the comprehensive naming rationale that I will investigate and formalize.

The following list presents a summary of my contributions:
\begin{enumerate*}
    \item a new, pattern-based approach that can detect and improve non-descriptive test names
    \item a novel, automated approach that can identify unique attributes of tests
    \item a name-generation approach to provide descriptive names based on a comprehensive naming rationale
\end{enumerate*}.
%
I anticipate that my research can systematically help test developers to detect non-descriptive test names and improve them using the unique attributes of tests.
%
Or developers could use the generated descriptive names to replace the non-descriptive ones, which increase the readability of their unit tests (i.e., consider tests as a kind of documentation).
%
At the same time, quality and maintainability of unit tests could be significantly improved by using my approaches.