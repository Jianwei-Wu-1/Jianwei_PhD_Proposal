\section{Foreseen Contributions}
\label{sec:contributions}

In this proposal, I propose a new approach to provide descriptive names for unit tests.
%
In the preliminary work, first, I developed a pattern-based approach to detect and improve non-descriptive test names.
%
Second, an empirical study to investigate if unit tests are named after what makes them unique was conducted to motivate the following work.
%
Based on the findings of the empirical study, I further proposed an automated approach to identify unique attributes of unit tests.
%
Finally, I plan to create a new name generation approach to provide descriptive test names that meet with develop approval, which would be built on top of both the unique attributes and some additional information.


The following list presents a summary of my contributions:
\begin{enumerate*}[label=(\roman*)]
    \item a novel, pattern-based approach that can detect and improve non-descriptive test names
    \item a new, name generation approach that can provide descriptive names
\end{enumerate*}.
%
In particular, the first piece of my dissertation work makes the following contributions:
\begin{enumerate*}
  \item A novel, pattern-based approach can detect non-descriptive test names of JUnit tests and provide descriptive information about the unit tests to help developers improve existing unit tests.
  \item A prototype implementation of the approach as an IntelliJ IDE plugin.
  \item An empirical evaluation on \num{10} Java projects that shows:
      \begin{enumerate*}[label=(\alph*)]
          \item the patterns are general and cover a majority of test names and bodies
          \item the patterns can accurately extract descriptive information from both test names and bodies
          \item the approach can accurately classify test names as either descriptive or non-descriptive
      \end{enumerate*}.
\end{enumerate*}
%
The second piece of my dissertation work makes the following contributions:
%
\begin{enumerate*}
    \item an empirical study to investigate if unit tests are named after what makes them unique
    \item an automated approach that uses the matchers of the selective codes and FCA to identify the unique attributes of unit tests
    \item a prototype implementation of the approach for JUnit tests
    \item an empirical evaluation to demonstrate the approach's output is consistent with human judgment
    \item a new name generation approach that can provide descriptive test names with developers approval
\end{enumerate*}.


I anticipate that my research can help developers both detect tests with non-descriptive names and generate descriptive names for them.
%
For example, developers could use the generated names to replace the non-descriptive names, which significantly increase the readability of their unit tests (i.e., consider unit tests as a part of software documentation).
%
At the same time, quality and maintainability of unit tests could be significantly improved by using my approaches.