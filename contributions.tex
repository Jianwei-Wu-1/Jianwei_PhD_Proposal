\section{Foreseen Contributions}
\label{sec:contributions}

In this proposal, I propose three research directions that focus on systematically providing descriptive and unique names for unit tests.
%
In the preliminary work, first, an empirical study to investigate if unit tests are named after what makes them unique was conducted to motivate all the following work.
%
Besides the preliminary work, I first plan to use the results from the empirical study to develop a concept-based approach that provides a uniqueness-based naming rationale for unit tests.
%
Then I plan to develop a pattern-based approach that can detect and improve non-descriptive test names, which is already completed in the preliminary work.
%
The empirical study and the pattern-based approach combine pattern mining, static program analysis, NLPA, selective coding, and quantitative analysis.
%
Secondly, I plan to create a fully automated framework for providing descriptive test names for existing Java projects by detecting non-descriptive test names and replacing them with descriptive names.

The following list presents a summary of my contributions:
\begin{enumerate}
    \item a new, pattern-based approach that can detect and improve non-descriptive test names
    \item a novel, concept-based approach that can generate descriptive test names by a uniqueness-based naming rationale
    \item a fully automated framework that can detect non-descriptive test names and replace them with automatically generated descriptive names
\end{enumerate}.

I anticipate that my research can systematically help test developers to identify and improve existing test names with three different types of descriptiveness that provide the summarized information from the test body.
%
At the same time, quality and maintainability of unit tests could be significantly improved by using my approaches.