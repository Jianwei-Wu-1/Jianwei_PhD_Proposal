\section{Related Work}
\label{sec:related-work}

Existing work related to my proposed research will be discussed in this section. I have organized them into the following categories:

\subsection{Detecting Mismatches\slash Improving Names}


There are some existing works that attempt to identify name\slash implementation mismatches.

\Citeauthor{host2009debugging}'s work is the most similar to our approach as it attempts to identify several types of naming bugs in general Java methods~\cite{host2009debugging}.
%
Their approach relies on a manually generated rule book that is extracted from the implicit convention between names and implementations in Java programming, which can be utilized to detect name bugs and provide some suggestions for constructing more suitable names.
%
In their previous works, \citeauthor{host2008java} already showed that there is a mutual dependency between method names and their associated implementations~\cite{host2008java}.
%
Therefore, their approach considered the mismatch between the name and the implementation of its associated method and used the mismatch to fix name bugs, which are both similar to the analytical process and goal of our pattern-based approach.
%
There are two major differences between their work and ours.
%
First, our approach primarily focuses on the test names rather than general method names that often follow a different naming convention.
%
For example, their approach treated the data type of the value in the \texttt{return} statement as an essential attribute in their rule book.
%
However, normally for the unit tests, they compared different values using the \texttt{assertions} rather than any \texttt{return} statement, so the information in those \texttt{assertions} will be a crucial part of their test names.
%
Second, instead of using a manually generated rule book, we built our approach based on the test patterns, and those test patterns were mined from a large test corpus by a semi-automatic process.

% TODO->[fixed, moved allamanis2015suggesting&daka2017generating to next section]: Are't some of these works more approach for generating names from scratch rahter than suggesting improvemnts?
\citeauthor{zhong2013detecting} provided a novel approach for detecting API documentation errors, and those errors are essentially the mismatches between the API documentation and the actual programs~\cite{zhong2013detecting}.
%
To address the importance of words in Java programming, \citeauthor{singer2008exploiting} showed that words in class names are closely related to class properties that can be described in micro patterns~\cite{singer2008exploiting}.
%
\citeauthor{allamanis2014learning} mentioned that developers should follow a consistent naming convention, and they proposed a novel framework that can suggest identifier names accurately~\cite{allamanis2014learning}.
%
All of their works comprehensively showed it is feasible to find poorly structured (i.e., non-descriptive) names by using the mismatch or pattern between the name and the program, and we can also improve those names by using providing accurate suggestions.
%
Nonetheless, each of their techniques is often limited to a certain aspect in the problem of detecting and improving non-descriptive names, so none of them can be directly applied to improving non-descriptive names in unit tests.
%
\citeauthor{pradel2018deepbugs} recently proposed a framework for the detection of naming bugs~\cite{pradel2018deepbugs}.
%
Regardless of their effort to introduce a new approach that can detect name-based bugs by using their machine learning method, we still can not apply their approach to the unit tests without further modifications.
%
Because some unit tests are expected to produce certain exceptions or failures when using them, so testers might intentionally design poorly named identifiers in those tests.
%
Consequently, lots of false-positives could be generated without a complete retrofit to extend their proposed framework to unit tests.
% change done ->
For instance, Junit 4 requires every test name to have a leading \enquote{test} \cite{JUnit4}, so some of existing techniques might consider the leading \enquote{test} as the action, predicate, or scenario of the name.


\subsection{Automated Generation of Test Names}

In contrast to the techniques mentioned above that attempt to improve names, there are several approaches that attempt to automatically generate names.
%
Some of these techniques use natural language-based program analysis~(NLPA).
%
For example, \citeauthor{zhang2016towards} proposed their approach that can generate descriptive names from existing test bodies by using natural-language program analysis and text generation~\cite{zhang2016towards}.
%
However, their approach left an important question unanswered that is testers need to decide which one of the three possible test names should be used for their unit tests by themselves, and it is possible that none of the three generated names follow the common naming convention.
%
Other techniques utilized Java bytecode, method-call sequences, API-level coverage goals, and \texttt{logbilinear} context models~\cite{fraser2011evosuite,thummalapenta2009mseqgen,daka2017generating,allamanis2015suggesting}.
%
Even with their automated generation process, their generated test names are not human-readable that can cause misunderstanding for testers who want to further modify those generated test names or bodies.
% change done ->
Although some techniques can generate descriptive names, those techniques required testers to perform a full test execution with certain coverage goals or building a context model, which are often error-prone in practice (i.e., those coverage goals or models might be too specific to apply for certain projects).

Recent work tried to solve the naming problem of unit tests by using machine learning-based approaches~\cite{alon2018code2seq, alon2019code2vec}.
%
Although using machine learning-based models can accurately summarize the important information from the test body, their predicted test names often do not fit developer need that is to be unique among its siblings.
%
To understand how limited these machine learning-based approaches are, we conducted a pilot study with a set of \num{11} randomly selected projects (i.e., contain \num{25567} unit tests) that contain a complete suite of JUnit tests.
%
The \num{11} Java projects are built for various purposes like Guava is an open-source common library for Java and fastjson is a JSON parser\slash generator.

If many generated test names are repeated at least once in its associated test class and there are lots of duplicated names during the name generation process, it is then necessary to develop a uniqueness-focused approach.
%
In order to measure the level of duplication, we performed a quantitative analysis on the collected result.
%
The data of the quantitative analysis is shown in this shared document~\cite{CodeResult}, and it shows that hundreds or thousands of duplicate test names were generated for each project (i.e., either using~\cite{alon2018code2seq} or~\cite{alon2019code2vec}).
%
For example, \enquote{Code2seq} produced \num{80} repeating test names for the moshi project, and \num{14} out of \num{24} test classes in the moshi project contain more than one repeating test names.
%
For another example, \enquote{Code2vec} produced \num{1199} repeating test names for the guava project, and \num{211} out of \num{461} test classes in the guava project contain more than one repeating test names.
%
This observation during the quantitative analysis indicates there is a need to develop a uniqueness-based naming rationale that can be used to generate descriptive (i.e., being unique and informative) test names.


\subsection{General Program Analysis\slash Automated Testing and Debugging}

Many researchers proposed their program analysis or automated testing techniques that can help us have a better understanding of the embedded features in unit tests. 
%
\citeauthor{moreno2012jstereocode} proposed Java method and class stereotypes, and they took a closer look at the statement level analysis of Java code~\cite{moreno2012jstereocode}, and~\citeauthor{ghafari2015automatically} tried to extract the focal method under test~\cite{ghafari2015automatically}.
%
To be more focused on unit testing,~\citeauthor{li2018aiding} constructed a series of tags for distinguish unit test cases~\cite{li2018aiding}.
%
A group of researchers also conducted a series of works related to tagging methods or classes with stereotypes~\cite{dragan2006reverse,dragan2010automatic,dragan2011emergent}, but their works might also not be applicable for unit tests.
%
From a general perspective of testing, other researchers tried to devise methods that can perform fully automated testing by the targeted event sequence or the environmental dependencies~\cite{jensen2013automated,arcuri2014automated}.
%
All of their works performed well under their specific problems in program analysis or automated testing. 
% change done here
However, while their works focus more on extracting features from code or automating the testing process rather than the unit tests themselves, we can still use them to improve our pattern-based approach.
%
Furthermore, \citeauthor{li2019deepfl} proposed a learning-based approach for fault localization and automated debugging with high performance~\cite{li2019deepfl}, but the goal of their work is primarily to locate software faults for debugging rather than the naming of unit tests. 
%
Regardless of the performance of their proposed tool, the goal of their work is primarily to locate software faults for debugging rather than the naming of unit testing, and the experimental subjects they used is a standard benchmark database of detecting bugs rather than the unit tests from real-world Java projects.


\subsection{Natural Language Program Analysis}

There are lots of existing works that try to analyze programs from a natural language-based perspective.
%
\citeauthor{pollock2007introducing} and~\citeauthor{shepherd2007using} introduced NLPA by illustrating how to apply NLPA in practice and also giving some insights about aspect mining~\cite{pollock2007introducing,shepherd2007using,pollock2009natural}.
%
Their studies showed natural language clues from developers' naming style can be used for improving the effectiveness of software tools.
%
\citeauthor{abebe2010natural} proposed a natural language-based method to parse the identifier names of program elements for extracting concepts from them~\cite{abebe2010natural}.
%
Furthermore, some researchers attempted to split identifiers~\cite{enslen2009mining,butler2011improving,guerrouj2013tidier,hill2014empirical}, and others managed to expand abbreviations~\cite{hill2008amap,madani2010recognizing,corazza2012linsen}.
%
Even though their works are not alternatives to our approach, we can still use their implemented tools to improve the accuracy of the test patterns.


\subsection{Formal Concept Analysis}

As a well-developed technique for deriving a concept hierarchy from objects through a set of attributes, formal concept analysis (FCA) is an excellent method to use for analyzing data like the process of human thinking~\cite{ganter2012formal,yao2004comparative}.
%
Moreover, a early survey has already been conducted to discover the possibility of solving software engineering problem by using concept analysis~\cite{tilley2005survey}.
%
Many researchers focused on using FCA as an efficient method to provide solutions for various software engineering problems~\cite{godin2005formal,park2000software,cellier2008formal,hesse2005formal,mens2005delving}, and they used concept analysis to mock how human (i.e., developers) thinks about writing software or trace the thinking inside existing software.
%
However, none of them focused on applying FCA to unit tests, and their technique paid more attention to use FCA to understand the high-level structure of software rather than improving the naming in unit tests.

Other works also managed to utilize formal concept analysis as a method to resolve specific problems.
%
\citeauthor{he2011test} developed FCA-based technique for test case generation of mutation testing \cite{he2011test}.
%
\citeauthor{khor2004using} proposed a automatic test generator that uses formal concept analysis to manage the relationships between tests \cite{khor2004using}.
%
\citeauthor{cimiano2005learning} tried to extract concept hierarchies from text corpus by using their proposed techniques based on formal concept analysis \cite{cimiano2005learning}.
%
\citeauthor{tonella2004aspect} managed to perform aspect mining on execution traces by using formal concept analysis \cite{tonella2004aspect}.
%
However, all of their works focused on solving those specific problems, so their approaches might not be applicable for providing descriptive names for unit tests.


\subsection{Investigating Developer's Focus in Software Development}

In recent years, a novel study of tracking the eye movement of developers emerged in the effort to understand the behavior of developers in the process of software development.
%
\citeauthor{rodeghero2015empirical} conducted an empirical study to discover the patterns of eye movement when developers perform summarization tasks~\cite{rodeghero2015empirical}.
%
\citeauthor{begel2018eye} performed a pilot study of capturing the eye movements of multiple developers in the process of code review, which was inspiring for building a automated tool for code review~\cite{begel2018eye}.
%
\citeauthor{abid2019using} focused on building a mental cognition model to understand how developers understand codes~\cite{abid2019using}.
%
\citeauthor{obaidellah2018survey} conducted a throughout survey on the usage of the eye-tracking technology in the research of program comprehension~\cite{obaidellah2018survey}.
%
\citeauthor{abid2019developer} presented a eye-tracking study that shows the behavior of developers when summarizing Java methods~\cite{abid2019developer}.
%
\citeauthor{ioannou2020mining} proposed a series of reading pattern by mining the eye-tracking data for comprehending the behavior of developers~\cite{ioannou2020mining}.
%
All of the recent work provided a great insight for us to choose which of the code elements should be selected as the secondary codes and helped us to have a profound understanding of the focus of actual developers.
%
Nevertheless, their work primarily focused on the eye-tracking of developers when reading the general Java methods and classes, not unit tests.
%
Moreover, their work did provide some results from those eye-tracking studies, but no formal approach is proposed to detect and extract the unique coding of unit tests.

