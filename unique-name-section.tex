\subsection{A Concept-based Approach to Improve Test Names with a Uniqueness based Naming Rationale}
\label{sec:unique-test-name}

In the first part of work, I consider the leading uniqueness-based approach to provide descriptive names, which is a concept-based approach that can provide a uniqueness-based naming rationale for constructing descriptive test names.
%
Before I start to build the concept-based approach, I first need to know whether unit tests are named after what makes them unique or not.
%
Therefore, an empirical study was conducted to check our thoughts and assumptions, and a set of top-level and secondary codes was formulated using selective coding~\cite{glaser1967discovery,strauss1998basics}.
%
Moreover, the collected results and naming rationales that were investigated in the empirical study will also be used to provide descriptive test names using other naming metrics.
%
The set of codes are used to tag the tests and extract the tagged text from those tests in the concept-based approach, which is capable of check if a test is named after what makes it unique and improve the uniqueness of that test name by utilizing the unique information from the tagged text.