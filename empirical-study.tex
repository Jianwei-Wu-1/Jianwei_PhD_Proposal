\subsection{An Empirical Study of Unit Test Naming Rationales}
\label{sec:emp-study}

As the first component towards generating descriptive test names, I conducted an empirical study to understand existing test naming rationales.
%
Our intuition is that developers often name unit tests based on what aspects of the test makes the test unique among its siblings.
%
To validate this assumption, we conducted an empirical study of \num{440} existing tests.
%
First, we examined each test in order to identify what makes it unique from its siblings.
%
Then we examined the test’s name and judged whether the name is based either entirely or in part on the unique aspect of the test.


\subsubsection{Study Methodology}

As the first part of this study, I randomly selected a set of \num{440} tests from \num{11} open-source projects from Github.
%
Moreover, I used an open, axial, and selective coding process to qualitatively analyze the tests.
%
Finally, I discussed the results of the coding process and the data to answer two research questions about whether unit tests are named, either wholly or in part, after what makes a given test unique among its siblings.

\paragraph{Experimental Subjects}


\paragraph{Code Creations}


\paragraph{Coding Process}


\subsubsection{Results and Discussion}

As the second part of this study...


\textbf{--Discussion--}


Based on my findings of this empirical study (e.g., tests are often named after what makes them unique), I proceeded to build an automated approach to extract unique attributes of unit tests.
%
Also, this study also showed more details about how developers would name their tests and motivated us to build the last component, which is a new descriptive name generation approach for unit tests.