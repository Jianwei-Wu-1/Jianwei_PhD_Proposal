\section{Automated Identification of Uniqueness for Generating Test Names}

In this section, first, I introduce a pilot study to motivate us investigate if tests are named after what makes them unique.
%
Second, I present an empirical study that accomplished that goal by looking into \num{440} randomly selected tests from Github.
%
Last, the automated approach is shown and explained in more detail.


\subsection{Duplication of Names Produced by Machine Learning-based Approaches}
\label{sec:duplication-names}

As we mentioned in~\cref{sec:introduction}, recent machine learning-based approaches produced a lot of duplication when generating test names.
%
To understand the limitation of those machine learning-based approaches, we cloned the top \num{50} Java projects from GitHub and randomly selected \num{11} projects that contain a complete suite of JUnit tests.
%
Then we analyzed all the unit tests from the \num{11} projects with those machine learning-based approaches and collected the generated test names from each project.
%
The \num{11} Java projects are built for various purposes like Guava is an open-source common library for Java and fastjson is a JSON parser\slash generator.
%
If there are many generated test names that are repeated at least once in its associated test class, it is necessary to develop a uniqueness-focused approach to solve the problem of duplication.
%
The collected results are shown in this shared document~\cite{CodeResult}.
%
It shows lots of duplicated test names were generated during the process of automated name generation. 
%
For example, \enquote{Code2seq} produced \num{80} repeating test names for the moshi project, and \num{14} out of \num{24} test classes in the moshi project contain more than one repeating test names.
%
For another example, \enquote{Code2vec} produced \num{1199} repeating test names for the guava project, and \num{211} out of \num{461} test classes in the guava project contain more than one repeating test names.
%
Obviously, using some of the state-of-the-art approaches that are based on machine learning still produce thousands of repeating test names when performing the automated name generation.
%
This result also indicates there is a need to develop a uniqueness-based naming rationale that can generate descriptive test names.


\subsection{Empirical Study: Are tests named for what makes them unique?}
\label{sec:empStudy}
\textbf{---Empirical Study---}

\subsection{Automated Approach to Identify Unique Attributes of Test}
\label{sec:auto-approach}
\textbf{---Automated Approach---}

\subsection{Empirical Evaluation}
\label{sec:emp-eval}
\textbf{---Empirical Evaluation---}



After the empirical study is completed and the automated approach for extracting unique attributes of unit tests is built, we are one step closer towards generating descriptive test names that meets with developer approval.
%
The following sections will introduce my research plan to:
\begin{enumerate*}
    \item perform a further investigation of the test names categorized as partially named after what makes the test unique
    \item combine the uniqueness-based naming rationale with minor ones to form a comprehensive naming rationale
    \item build a name-generation tool to generate descriptive test names based on the comprehensive naming rationale 
\end{enumerate*}

