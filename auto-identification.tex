\subsection{Automated Identification of Uniqueness for Generating Test Names}
\label{sec:uniquness-approach}

As the second step towards generating descriptive test names, I built an automated approach to extract unique attributes of unit tests based on the results of previous step.
%
This part of work is combined with the empirical study and submitted to ACM Transactions on Software Engineering and Methodology (Currently under administrative processing).
%
To check the consistency of the approach, I further conducted an empirical evaluation on a set of randomly chosen tests from various open-source projects on Github by using a working prototype of it.
%
The conclusion of the evaluation shows the approach is consistent with human judgment and useful for future name generation techniques.


\subsection{Automated Approach to Identify Unique Attributes of Test}
\label{sec:auto-approach}
\textbf{---Automated Approach---}

\subsection{Empirical Evaluation}
\label{sec:emp-eval}
\textbf{---Empirical Evaluation---}


After the empirical study is completed and the automated approach for extracting unique attributes of unit tests is built, we are one step closer towards generating descriptive test names that meets with developer approval.
%
The following sections will introduce my future research plan to:
\begin{enumerate*}
    \item perform a further investigation of the test names categorized as partially named after what makes the test unique
    \item combine the uniqueness-based naming rationale with minor ones to form a comprehensive naming rationale
    \item build a name generation tool to generate descriptive test names based on the comprehensive naming rationale 
\end{enumerate*}

