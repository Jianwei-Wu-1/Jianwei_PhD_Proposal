\section{Proposed Work}
 
 %TODO[done, change for all]: This title doesn't really make sense.  It's too long.  It's also too focused on concept-based.  That's an implementation detail at this point.  The uniquness-based naming approach is what is important.

\subsection{Improve and Construct Descriptive Names for Existing Unit Tests by the Uniqueness-based Naming Rationale}
\label{sec:unique-test-name}

For the second part of work, I consider using a uniqueness-based test naming approach to automatically generate descriptive names, which is accomplished by building an implementation with a set of selective codes from the empirical study~\cite{emp-study}.
%
The empirical study was conducted to check our thoughts and assumptions, so a set of top-level and secondary codes can be formulated using selective coding~\cite{glaser1967discovery,strauss1998basics}.
%
The set of codes are used to tag the tests and extract the tagged text from those tests in the concept-based approach, which is capable of check if a test is named after what makes it unique and improve the uniqueness of that test name by utilizing the unique information from the tagged text.

\subsection{An Fully Automated Framework to Provide Descriptive Test Names for Developers}

For the last part of work, a further evaluation of both the pattern-based and uniqueness-based approaches will be conducted as a pilot study for an automated framework to replace non-descriptive test names.
%
An developer-oriented survey will also be conducted with a group of seasoned developers, and they will be asked to describe how they write their unit tests and how they modify the existing tests.
%
Based on the developer-oriented survey, we could have a better understanding of how developers name their tests.
%
After our researchers would gain a general sense about when to apply the pattern-based approach to detect non-descriptive test names and how to apply the uniqueness-based approach to generate descriptive names, the automated framework can be built based on the pilot study and developer-oriented survey.



