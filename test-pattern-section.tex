\section{A Pattern-based Approach to Detect and Improve Non-descriptive Test Names}
\label{sec:test-pattern-section}

For the first piece of my dissertation work, I considered a pattern-based approach that can help developers identify and improve non-descriptive test names using the extracted descriptive information from the tests.
%
This piece of work is published on Journal of System and Software~\cite{wu2020pattern}.
%
The detection of non-descriptive test names is to extract descriptive information from the test name and body and use the extracted information  from the name and body to conduct an information comparison to detect non-descriptive names.
%
This focus of this piece of work was summarizing the descriptive information from the test body and compare it with the extracted information from the test name.
%
As developers have the ability to know if a test name is descriptive or non-descriptive, they can further utilize several suggestions that are generated by the information comparison of the matched name and body patterns to improve those non-descriptive test names.


\subsection{Identification of Test Patterns}
\label{sec:test_patterns}

\subsection{Description of the Approach}
\label{sec:approach-pattern}

\subsection{Evaluation}
\label{sec:evaluation-pattern}


With the knowledge of how to identify and improve non-descriptive names, I moved on to build the second piece that is to provide descriptive names for unit tests.
%
An empirical study was conducted to investigate the unit test naming rationales.
%
Moreover, an automated approach was developed to extract the unique attributes of tests and used as the starting point to generate descriptive test names.