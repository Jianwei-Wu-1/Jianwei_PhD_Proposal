\section{A Pattern-based Approach to Detect and Improve Non-descriptive Test Names}
\label{sec:test-pattern-section}

As the first piece my dissertation work, I developed a pattern-based approach that can help developers identify and improve non-descriptive test names.
%
This piece of work is published in the Journal of System and Software~\cite{wu2020pattern}.

% TODO: Note how this is part of the introduction to the patterns paper.  Some parts of it's intro were moved to the proposal's intro, but the rest of it needs to go here.  We still need to have a introduction to this piece of work.
The pattern-based approach can:
\begin{enumerate*}
\item detect non-descriptive test names by finding mismatches between the name and body of a given JUnit test
\item provide descriptive information that consists of the main motive of test, the property to be tested in the test, and the prerequisite needed in the test or the object to be tested (see~\cref{sec:test_patterns} for details) to facilitate the improvement of non-descriptive test names
\end{enumerate*}.
%
Unlike existing approaches that suggesting improvements, which were designed to handle general methods, our approach is specific to JUnit tests.
% TODO: Make sure cref points to the right place...
The narrower scope of the work allows it to take advantage of the highly repetitive structures that exist in both test names and bodies of JUnit tests (see~\cref{sec:test_patterns}).
%
From a high-level point of view, the approach uses a set of predefined patterns to extract descriptive information from both a test's name and body.
%
This information is then compared to find non-descriptive names (i.e., cases where the name does not accurately summarize the body).
%
When a mismatch is found, the information used by the approach can help developers address the mismatch and improve the quality of the test name.


To assess the pattern-based approach, we implemented it as an IntelliJ IDE plugin.
%
The plugin was then used to carry out an empirical evaluation of the quality of more than \num{34000} tests from \num{10} Java projects.
%
Overall, the results of our evaluation are promising and show that the pattern-based approach is feasible, accurate, and effective.

% TODO: Maybe this should be moved to be part of the contributions section...
In particular, this piece of my dissertation work makes the following contributions:
\begin{itemize}
  \item A novel, pattern-based approach can detect non-descriptive test names of JUnit tests and provide descriptive information about the unit tests to help developers improve existing unit tests.
  \item A prototype implementation of the approach as an IntelliJ IDE plugin.
  \item An empirical evaluation on \num{10} Java projects that shows:
      \begin{enumerate*}
          \item the patterns are general and cover a majority of test names and bodies
          \item the patterns can accurately extract descriptive information from both test names and bodies
          \item the approach can accurately classify test names as either descriptive or non-descriptive
      \end{enumerate*}.
\end{itemize}


\subsection{Identification of Test Patterns}
\label{sec:test_patterns}

\subsection{Description of the Approach}
\label{sec:approach-pattern}

\subsection{Evaluation}
\label{sec:evaluation-pattern}
