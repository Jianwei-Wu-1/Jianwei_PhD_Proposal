\section{A Pattern-based Approach to Detect and Improve Non-descriptive Test Names}
\label{sec:test-pattern-section}

For the first piece of my dissertation work, I considered a pattern-based approach that can help developers identify and improve non-descriptive test names using the extracted descriptive information from the tests.
%
This piece of work is published on Journal of System and Software~\cite{wu2020pattern}.
%
The detection of non-descriptive test names is to extract descriptive information from the test name and body and use the extracted information  from the name and body to conduct an information comparison to detect non-descriptive names.
%
This focus of this piece of work was summarizing the descriptive information from the test body and compare it with the extracted information from the test name.
%
As developers have the ability to know if a test name is descriptive or non-descriptive, they can further utilize several suggestions that are generated by the information comparison of the matched name and body patterns to improve those non-descriptive test names.


\subsection{Test Patterns}
\label{sec:test_patterns}

\subsection{Description of the Pattern-based Approach}
\label{sec:approach-pattern}

\subsection{Empirical Evaluation of the Pattern-based Approach}
\label{sec:evaluation-pattern}


After acknowledging how to identify and improve non-descriptive names, I moved on to the second piece of providing descriptive test name.
%
First, an empirical study was conducted to investigate if unit tests are named after what makes them unique as the primary motivation.
%
If so, I planned to further develop an automated approach to extract what makes a given test unique and use it as the starting point to generate descriptive test names.
