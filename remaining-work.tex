\section{Remaining Work: A systematic Approach to Provide Descriptive Test Names}
\label{sec:remaining-work}

In the remaining work, I plan to develop a systematic name-generation approach to provide descriptive test names in developers' favor.
%
First, I will look into the minor naming rationales found in~\cref{sec:empStudy} and the test names that were categorized as partially named after what makes it unique in detail.
%
After understanding where all of the information in developers' test names originate from, it would be possible for me to build a comprehensive naming rationale based on that.
%
Second, using the comprehensive naming rationale, I will build a name-generation approach to generate descriptive test names for unit tests, which is in the same naming rationale as developers have.
%
Last, as mentioned in~\cref{sec:introduction}, the goal of my proposed research is to systematicly provide descriptive test names for developers, so I plan to evaluate the generated descriptive names by their coverage, accuracy, and consistency.


\subsection{A systematic Approach to Provide Descriptive Test Names}

As the last part of the my dissertation work, I plan to propose a systematic approach to provide descriptive test names.
%
For the first step of the exploratory study mentioned in~\cref{sec:introduction}, a further investigation of the tests that were categorized as partially named after what makes them unique.
%
The goal of this step is to find the origin of other information embedded in descriptive names and check if it is possible to formalize a naming rationale to extract that information.
%
For the second step of the exploratory study, I plan to look into the minor naming rationales discovered in the empirical study.
%
Combining the other information in names with minor naming rationales, a comprehensive naming rationale can be formed towards generating descriptive test names.
%
Finally, a novel, systematic name-generation approach is going to be developed to provide descriptive test names for unit test.
%
Moreover, its generated names are approved by developers, which will significantly reduce the burden of test maintenance for them.
%
This approach is designed to be a fully automated process, so there is no human input needed (i.e., neither to detect poor names nor generate descriptive names).


\subsection{Evaluation of the systematic Name-generation Approach}

The planned research questions in this step are:
%
\begin{enumerate}
    \item RQ1-Coverage: How often can the systematic approach generate descriptive names for tests that are poorly named?
    \item RQ2-Accuracy: Are the generated test names by the systematic approach descriptive about the corresponding test's unique and other attributes?
    \item RQ3-Consistency: Do the the generated test names by the systematic approach have developer approval?
\end{enumerate}.

For the experimental subjects, I plan continue to use the \num{11} project that were studies in the empirical study.
%
To answer RQ1, I plan to perform a quantitative analysis by counting the number of poor names per project.
%
After running my naming-generation approach on these subjects, the counting is compared to the number of generated names for those tests that are poorly named.
%
Judging from the results of RQ1, my approach should be able to those tests that are poorly named and provide descriptive names for them.
%
To answer RQ2, I plan to conduct an empirical evaluation that compares the generated names with human judgement.
%
To distinguish from RQ3, the human judgement for RQ2 is the manually extracted unique and other attributes of the tests.
%
This empirical evaluation targets to check if the generated names contain the same information that could be extracted by human.
%
To answer RQ3, I plan to conduct another empirical evaluation with a group of experienced developers (i.e., senior engineers with more than \num{3} years of experience) by asking them if the generated names are better than the original names, and:
\begin{enumerate*}
    \item if yes, at which category would you put each generated name in?
    \item if no, how would you name the corresponding tests?
\end{enumerate*}.
%
Based on our observation and assumption, the systematic name-generation approach should provide descriptive test names for JUnit test, which are considered and constructed in the same way as developers would do. 


\subsection{Research Plan}

The research plan of all proposed work contains three steps.
%
The first step has been done in my preliminary work that is to build a pattern-based approach to detect and improve non-descriptive test names.
%
The second step has been done in my preliminary work that is to build an automated approach to identify unique attributes of unit tests.
%
The following steps will be our planned future work.
%
The third step is to develop a systematic name-generation approach that can provide descriptive names for poorly named tests.
%
In order to build this approach, I need to take a further look at the tests that were categorized as partially named after what makes them unique and those minor naming rationales.
%
After the approach is successfully built and implemented, I plan to evaluate the systematic name-generation approach with three research questions, which will show us the coverage, accuracy, and consistency of my approach.