\subsection{Remaining Work: A New Approach to Provide Descriptive Test Names}
\label{sec:remaining-work}

As the last component towards generating descriptive test names, I plan to develop a new name generation approach to provide descriptive test names that match well with developer's cognition.
%
First, I will look into the test names that were categorized as partially named after what makes it unique in detail.
%
After understanding where all of the information in developers' test names originate from, it would be possible for me to build a comprehensive naming rationale based on that.
%
Second, using the comprehensive naming rationale, I will build a new name generation approach to generate descriptive test names for unit tests, which uses the same naming rationale as developers would do.
%
Last, the goal of my proposed research is to provide descriptive test names for developers, so I plan to evaluate my approach's generated names by their coverage and effectiveness.


For the first step of the exploratory study, a further investigation of the tests that were categorized as partially named after what makes them unique.
%
The goal of this step is to find the origin of other information embedded in descriptive names and check if it is possible to formalize a naming rationale to extract that information.
%
For the second step of the exploratory study, I will combine the uniqueness-based naming rationale with the minor naming rationales as a comprehensive naming rationale.
%
This comprehensive naming rationale should contain all necessary attributes to name a given test in the same way as developers do.
%
Finally, based on the comprehensive naming rationale, a new, test name generation approach is going to be developed to provide descriptive test names for unit test.
%
I also plan to figure out an optimal solution to concatenate the necessary attributes to provide descriptive names that would be preferred by developers.
%
Moreover, since my approach's generated names are approved by developers, they will significantly reduce the time and effort of test maintenance and documentation of the related software.
%
This approach is designed to be a fully automated process, so there is no human input needed (i.e., neither to detect poor names nor generate descriptive names).


\subsubsection{Evaluation of the New Test Name Generation Approach}

The planned research questions in this step are:
%
\begin{enumerate}
    \item RQ1-Coverage: How often can the approach generate descriptive names for tests that are poorly named?
    \item RQ2-Effectiveness: Do the generated test names by the approach have developer approval?
\end{enumerate}
%
For the experimental subjects, I continue to use the same \num{6} projects that were studies in~\cref{sec:emp-eval-attributes}.
%
To answer RQ1, I plan to perform a quantitative analysis by calculating the number of poorly named tests per project.
%
After running the naming generation approach on these subjects, my approach should be able to cover the poorly named tests and provide descriptive names for them.
%
To answer RQ2, I plan to conduct an empirical survey with a group of experienced developers (i.e., senior engineers with more than \num{3} years of experience) by asking them if the generated names are better than the original names, and:
\begin{enumerate*}
    \item if yes, at which category would you put each generated name in?
    \item if no, how would you name the corresponding tests?
\end{enumerate*}.
%
Based on our observation and assumption, the new name generation approach should be able to provide descriptive test names that are constructed in the same way as developers would do. 

\subsubsection{Research Plan}

The research plan of all proposed work contains two pieces.
%
The first piece has been done in my preliminary work that is to build a pattern-based approach to detect and improve non-descriptive test names.
%
The second piece is develop a new name generation approach that can provide descriptive names for poorly named tests.
%
This piece contains three components.
%
The first component has been done to conduct an empirical study to investigate if tests are named after what makes them unique, which is completed in my preliminary work.
%
The second component has been done to to build an automated approach to identify unique attributes of unit tests.
%
The last component is my planned future work, which is to develop a new name generation approach for unit tests.
%
In order to build this approach, I need to take a further look at the tests that were categorized as partially named after what makes them unique and those minor naming rationales.
%
After the approach is successfully built and implemented, I plan to evaluate the new name generation approach with three research questions, which will show us the its coverage and effectiveness.