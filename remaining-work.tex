\section{Remaining Work: Implementation and Evaluation of the Automated Descriptive Name Generation Approach}
\label{sec:remaining-work}

In the remaining work, I plan to develop a uniqueness-based approach to generate descriptive test names.
%
First, using the previously selected set of top-level and secondary codes from the empirical study, it is feasible for me to build a corresponding set of matchers for the selective codes.
%
Second, after the matchers are built, I plan to complete the uniqueness-based approach using formal concept analysis.
%
Last, as mentioned in~\cref{sec:introduction}, the goal of my proposed research is to systematically provide descriptive test names for developers (i.e., they are based on what makes the test unique), so I plan to evaluate if the generated names are actually more descriptive than existing test names.

\subsection{A Uniqueness-based Approach to Provide Descriptive Test Names}


Utilizing the top-level and secondary codes that were created in the empirical study section~\cref{sec:empStudy}, it is feasible to build an automated approach that generate descriptive test names from a uniqueness-based naming rationale.
%
For the implementation, I plan to introduce the idea of formal concept analysis (FCA) as the main method and show how to integrate it with the top-level and secondary codes.
%
Given an unit test, the uniqueness-based approach would first try and match the applicable selective codes for the test and use their corresponding matchers to extract attributes from the test.
%
Using FCA, the approach automatically checks the uniqueness of tests and generates a descriptive test name for the given test if necessary.


\subsection{Evaluation of the Automated Descriptive Name Generation Approach}


Research questions involved in this step are:
%
\begin{enumerate}
    \item RQ1-Consistency: Are the automatically matched codes by the uniqueness-based approach consistent with the manual tagging by our researchers?
    \item RQ2-Descriptiveness: Are the generated test names by the uniqueness-based approach descriptive?
\end{enumerate}.

To answer RQ1, I plan to perform a quantitative analysis by comparing the manual tagging of a set of \num{240} tests from \num{6} randomly selected Java projects on Github with the automated tagging of the uniqueness-based approach.
%
To answer RQ2, I plan to conduct an empirical evaluation with \num{6} experienced developers by letting them choose between the original and the generated names for the  \num{240} tests used for evaluating the approach.


