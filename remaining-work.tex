\subsection{Remaining Work: A New Approach to Provide Descriptive Test Names}
\label{sec:remaining-work}

As the last component towards generating descriptive test names, I plan to develop a new name generation approach to provide descriptive test names that match well with developers cognition.
%
First, I will look into the test names that were categorized as partially named after what makes it unique in detail.
%
With the knowledge of where other information in existing test names originate from, it would be possible to include some additional attributes in the descriptive name generation process.
%
Second, using both the unique and additional attributes, I will build a new name generation approach to generate descriptive test names.
%
To achieve that, I plan to figure out the complex transformation process that produces descriptive names and use it as an essential part of the name generation process.
%
Based on the the transformation from the attributes to descriptive names, I will build an automated approach for descriptive name generation.
%
Last, the goal of my proposed research is to provide descriptive test names for developers, so I plan to evaluate my approach's generated names by their coverage and effectiveness.


\subsubsection{Research Plan}

The research plan of my dissertation work is described below.
%
For the first step of building the name generation approach, a further investigation of the tests that were categorized as partially named after what makes them unique.
%
The goal of this step is to find the origin of additional attributes embedded in descriptive names and check if it is possible to figure out an automated way to extract them from tests.
%
For the second step of building the name generation approach, I plan to understand the complex transformation process from the unique and additional attributes to descriptive test names.
%
Moreover, I plan to figure out an optimal solution to transform the necessary attributes (e.g., add descriptive words, remove repetitive words, or modify certain phrases) to descriptive names that would be preferred by developers.
%
Last, based on the complex transformation process, a new test name generation approach would be automated and implemented to provide descriptive names for unit test.
%
Because my approach's generated names are produced with developers approval, they will significantly reduce the time and effort of test maintenance and documentation of the related software.
%
This approach is designed to be a fully automated process, so there is no human input needed (i.e., neither to detect poor names nor generate descriptive names).
%
Using the implementation of the approach, I plan to evaluate it with two research questions that are checking its coverage and effectiveness.


The planned research questions in this step are:
%
\begin{enumerate}
    \item RQ1-Coverage: How often can the approach generate descriptive names for tests that are poorly named?
    \item RQ2-Effectiveness: Do the generated test names by the approach have developers approval?
\end{enumerate}
%
For the experimental subjects, I plan to use the same \num{6} projects that were studies in~\cref{sec:emp-eval-attributes}.
%
To answer RQ1, I plan to perform a quantitative analysis by calculating how often the approach can produce a descriptive name for the poorly named tests.
%
Based on our preliminary observation, my approach should be able to cover most of the poorly named tests and provide descriptive names for them.
%
To answer RQ2, I plan to conduct an empirical survey with a group of experienced developers (i.e., software engineers in real-world with more than \num{3} years of experience) by asking them if the generated names are descriptive about their corresponding tests, and
\begin{enumerate*}
    \item if yes, at which category would you put each generated name in?
    \item if no, how would you name the corresponding tests?
\end{enumerate*}
%
Based the results of the empirical study in~\cref{sec:emp-study}, the new name generation approach should be able to provide descriptive test names that are constructed in the same way as developers would do.


\subsubsection{Proposed Timeline}

Currently, I already published the first piece of my dissertation work, which is a pattern-based approach to detect and improve non-descriptive test names~\cite{wu2020pattern}.
%
I also completed the first and second components of the second piece of my dissertation work, and they are combined and submitted to ACM Transactions on Software Engineering and Methodology.
%
For my planned future work, I will start working on the last component of the second piece during Fall 2021 and complete this part by the end of Winter 2022.
%
Finally, my PhD dissertation defense is going to take place during Spring 2022.