\subsubsection{Remaining Work: Implementation and Evaluation of the Concept-based Approach}
\label{sec:remaining-work}

% In my preliminary work, I built a pattern-based approach to detect and improve non-descriptive test names.
% %
% However, the pattern-based approach is limited to one perspective of the solving the naming problem for unit tests that is to improve test names with descriptiveness, but the uniqueness of tests is not mentioned in this work.
% %
% Especially, pattern mining in the pattern-based approach requires close attention to each component of every test pattern, which is both difficult and labor intensive.

In the remaining work, I plan to develop a concept-based approach that uses a set of top-level and secondary codes produced by the selective coding in the empirical study for tagging and extracting the uniqueness from the tests.
%
First, using the set of top-level and secondary codes that were previously selected, it is feasible for us to build a concept-based approach to check and improve existing test names with a uniqueness-based naming rationale.
%
Second, after the concept-based approach is completed, an empirical evaluation is also needed to further investigate the correctness, effectiveness, and comprehensibility of the approach.
%
As mentioned in~\cref{sec:introduction}, the goal of my proposed research is to systematically provide descriptive and unique test names, so there are still several attributes of test names that can be explored.
%
Therefore, for the last aspect of test names, I plan to consider the naming problem of unit tests from another innovative aspect, which is to build a shape-based approach that learn from how different shapes of a test name (i.e., length, camel cases, underscores, descriptiveness, uniqueness) affect developers' understanding of the corresponding tests.

\subsubsection{Implementation of the Concept-based Approach}

Utilizing the top-level and secondary codes that were created in the empirical study section~\cref{sec:empStudy}, the next step is to build a concept-based approach that can improve existing test names by a uniqueness-based naming rationale for developers.
%
For the implementation, I plan to introduce the idea of formal concept analysis (FCA) as the main method, and how to integrate it with the top-level and secondary codes will also be explained.
%
Using the implicit relationships extracted by FCA and the existing codes, a concept-based approach can not only correctly tag the codes to the tests but also provide a uniqueness-based naming rationale for improving the existing test names.

\subsubsection{Evaluation of the Concept-based Approach}

Research questions involved in this step are:
%
\begin{enumerate}
    \item RQ1-Correctness: Can the concept-based approach apply the top-level and secondary codes to each test correctly?
    \item RQ2-Effectiveness: Does the tagged text of each test accurately describe the uniqueness of the test?
    \item RQ3-Comprehensibility: Is the generated results of the concept-based approach straightforward enough to be easily understood?
\end{enumerate}.

For answering RQ1, I plan to perform a quantitative analysis by comparing the results from manual tagging of a randomly set of tests (i.e., from 5 random Java projects on Github) with the results from the concept-based approach to see if the two different results are matched.
%
For answering RQ2, I plan to further check if the tagged text that was generated by the concept-based approach is the same as the tagged text from the manual tagging from the same set of tests.
%
For answering RQ3, I plan to conduct a small scale survey that 4 experienced developer will participate, and several survey questions about if they can easily understand and apply the results from running the concept-based approach will be presented to the participants.


