\documentclass[proposal.tex]{subfiles}
\begin{document}

\newpage
\section{Introduction}
\label{sec:introduction}


Unit tests are an important artifact that supports the software development process in several ways.
%
In addition to helping developers ensure the quality of their software by checking for failures~\cite{daka2014survey}, they can also serve as an important source of documentation.
%
For example, when a test fails, its name can provide the first step towards understanding the purpose of the test and ultimately fixing the cause of the observed failure.
%
Similarity, a test's name can help developers decide whether a test should be left alone, modified, or removed in response to changes in the application under test and whether the test should be included in a regression test suite.


In this proposal, I believe that test names in unit tests are \enquote{good} if they are descriptive (i.e., they accurately summarize both the scenario and the expected outcome of the test~\cite{trenk14}) and \enquote{bad} if they are not descriptive.
%
This is because descriptive names:
\begin{enumerate*}
\item make it easier to tell if some functionality is not being tested---if a behavior is not mentioned in the name of a test, then the behavior is not being tested
\item help prevent tests that are too large or contain unrelated assertions---if a test cannot be summarized, it likely should be split into multiple tests
\item serve as documentation for the class under test---a class's supported functionality can be identified by reading the names of its tests
\end{enumerate*}~\cite{zhang2015automatically}.
%
Furthermore, I expand the idea of descriptive names to a global scope.
%
Rather than limiting to locally descriptive names, I start to investigate if we can provide test names that are not only descriptive but also global.
%
The word \enquote{global} means that a test name should not be derived from it related unit test solely, and it need to be considered from a larger scope (i.e., project level).
%
By using the larger scope, my working prototype of this proposal will provide descriptive test names that can also be globally recognized.


Unfortunately, unit tests often lack descriptive names.
%
For example, an exploratory study by \citeauthor{zhang2015automatically} found that only \SI{9}{\percent} of the \num{213423} test names they considered were complete (i.e., fully described the body of test) while \SI{62}{\percent} were missing some information and \SI{29}{\percent} contained no useful information (e.g., tests named \enquote{test})~\cite{zhang2015automatically}.
%
Poor test names can be due to developers writing non-descriptive or incomplete names.
%
They can also occur due to incomplete code modifications.
%
For example, a developer may modify a test's body but fail to make the corresponding changes to the test's name.
%
Regardless of the cause, non-descriptive test names complicate comprehension tasks and increase the costs and difficulty of software development.


Because non-descriptive names negatively impact software development, there have been several attempts to address this issue.
%
One approach has been to automatically generate names based on implementations (e.g.,~\cite{arcuri2014automated, zhang2015automatically, daka2017generating}).
%
For example, \citeauthor{zhang2015automatically} and \citeauthor{daka2017generating} use static and dynamic analysis, respectively, to extract important expressions from a test's body and natural language processing techniques to transform such expression into test names \cite{zhang2015automatically, daka2017generating}. 
%
While automatically generating names from bodies eliminates the possibility of mismatches between names and bodies the generated names do not always meet with developer approval (e.g., they may not fit with existing naming conventions).
%
Another approach is to help developers improve their existing names by suggesting improvements.
%
For example, \citeauthor{host2009debugging} proposed an approach for Java methods and variables which uses a set of naming rules and related semantics~\cite{host2009debugging} and \citeauthor{allamanis2015suggesting} and \citeauthor{pradel2018deepbugs} use a model-based and a learning-based approach, respectively, to directly suggest better names or find name-based bugs to facilitate improvements~\cite{allamanis2015suggesting, pradel2018deepbugs}.
%
In my dissertation work, I will focus on systematically providing descriptive names for unit tests and expanding the same idea to all names in unit tests (e.g., test class names, identifier names, etc).
%
More specifically, I propose the following research tasks:


First, I investigated is about improving non-descriptive test names by using a set of mined test patterns.
%
The test pattern are mined from a large corpus of unit tests (i.e., both names and bodies), and I subtracted the commonalities between the names and bodies as test patterns.
%
Applying the test patterns to the unit tests, we can extract the descriptive information from the tests, and compare the extracted information to the existing test name.
%
From the comparison between the information from body and from name, we can detect non-descriptive names and provide suggestions to fix them.


Second, I propose a concept-based technique to globally provide descriptive test names since the naming of tests is not \enquote{local}.


Finally, I will develop is a comprehensive study that can provide descriptive names for every component in unit tests.


\end{document}