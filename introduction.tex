\section{Introduction}
\label{sec:introduction}

Unit tests are an important artifact that supports the software development process in several ways.
% change done here 
In addition to helping developers ensure the quality of their software by checking for failures~\cite{daka2014survey}, they can also serve as an important source of documentation not only for human developers but also for automated software engineering tools (e.g., recent work on fault localization by \citeauthor{li2019deepfl} uses test name information~\cite{li2019deepfl}).
%
For example, when a test fails, its name can provide the first step towards understanding the purpose of the test and ultimately fixing the cause of the observed failure.
%
Similarly, a test's name can help developers decide whether a test should be left alone, modified, or removed in response to changes in the application under test and whether the test should be included in a regression test suite.


In this work, we believe that test names are \enquote{good} if they are descriptive (e.g, they accurately summarize both the scenario and the expected outcome of the test~\cite{trenk14}, or being unique among its associated test class) and \enquote{bad} if they are not descriptive.
%
This is because descriptive names:
\begin{enumerate*}
\item make it easier to tell if some functionality is not being tested---if a behavior is not mentioned in the name of a test, then the behavior is not being tested
\item help prevent tests that are too large or contain unrelated assertions---if a test cannot be summarized, it likely should be split into multiple tests
\item serve as documentation for the class under test---a class's supported functionality can be identified by reading the names of its tests
\end{enumerate*}~\cite{zhang2015automatically}.


Unfortunately, unit tests often lack descriptive names.
%
For example, an exploratory study by \citeauthor{zhang2015automatically} found that only \SI{9}{\percent} of the \num{213423} test names they considered were complete (i.e., fully described the body of test) while \SI{62}{\percent} were missing some information and \SI{29}{\percent} contained no useful information (e.g., tests named \enquote{test})~\cite{zhang2015automatically}.
%
Poor test names can be due to developers writing non-descriptive or incomplete names.
%
They can also occur due to incomplete code modifications.
%
For example, a developer may modify a test's body but fail to make the corresponding changes to the test's name.
%
Regardless of the cause, non-descriptive test names complicate comprehension tasks and increase the costs and difficulty of software development.


Because non-descriptive names negatively impact software development, there have been several attempts to address this issue.
%
One approach has been to automatically generate names based on implementations (e.g.,~\cite{arcuri2014automated, zhang2015automatically, daka2017generating}).
%
For example, \citeauthor{zhang2015automatically} and \citeauthor{daka2017generating} use static and dynamic analysis, respectively, to extract important expressions from a test's body and natural language processing techniques to transform such expression into test names \cite{zhang2015automatically, daka2017generating}. 
%
While automatically generating names from bodies eliminates the possibility of mismatches between names and bodies the generated names do not always meet with developer approval (e.g., they may not fit with existing naming conventions).
%
Another approach is to help developers improve their existing names by suggesting improvements.
%
For example, \citeauthor{host2009debugging} proposed an approach for Java methods and variables which uses a set of naming rules and related semantics~\cite{host2009debugging}, \citeauthor{li2019deepfl} provided a learning-based approach to locate software faults using test name information \cite{li2019deepfl}, and \citeauthor{allamanis2015suggesting} and \citeauthor{pradel2018deepbugs} use a model-based and a learning-based approach, respectively, to directly suggest better names or find name-based bugs to facilitate improvements~\cite{allamanis2015suggesting, pradel2018deepbugs}.


To address the lack of descriptive names in unit tests (i.e., the naming problem), I propose a series of approaches that can systematically provide descriptive names for unit tests.
%
The focus of all the approaches is to provide descriptive names (i.e., \enquote{good} names) for unit tests.
%
However, from the collected results of the first empirical study~\cref{sec:empStudy}, different developers often use different naming rationales for writing test names.
%
So we decide to develop multiple approaches that can solve the naming problem from three different naming rationales (i.e., three different perspectives of providing \enquote{good} names).
%
The three different naming rationales are: uniqueness-based naming, descriptiveness-based naming, and shape-based naming.
%
Any test name that is named after at least one of the three naming rationales will be considered as a descriptive test name, which also means that a \enquote{good} test name could be named after multiple naming rationales.


In this proposal, I propose the following research tasks, which will contribute to three approaches that can systematically provide descriptive naming from three different perspectives.
%
First, in order to show the evidence that different developers often use different naming rationales, an empirical study is conducted as the first step of building the three approaches, and a set of selective codes are also produced as part of the results of the empirical study.
%
Using the results of the empirical study, we found out that most of the inspected tests are indeed named after what makes them unique, and the uniqueness of tests can be identified and utilized for future improvement by a set of top-level and secondary codes, which comes from the selective coding process using basic Java code elements as shown in~\cref{sec:test-pattern-section}.
%
Moreover, different test names are shown to be named after different naming rationale.
%
Some tests are named after the descriptiveness of the test name and body, while other tests are named after the uniqueness of the test among it associated test class or the shape of test.
%
As the first perspective of the descriptiveness of test names, I propose a novel, concept-based approach that can:
%
\begin{enumerate*}
\item verify whether a test is named after what makes it unique or not
\item utilize the uniqueness of tests from the tagged text of tests to provide a uniqueness-based naming rationale for renaming unit tests
\item generate straightforward results using formal concept analysis (FCA)~\cite{ganter2012formal}
\end{enumerate*}.
%
Unlike existing approaches that were designed to handle general methods, our approach is not only specific to JUnit tests but also one of the pioneers to focus on providing a uniqueness-based naming rationale to help developers improve existing test names.
%
From a high-level point of view, the concept-based approach uses a set of selective code for tagging unit tests that can also provide unique information about the test.


Second, I propose a new, pattern-based approach that can:
\begin{enumerate*}
\item detect non-descriptive test names by finding mismatches between the name and body of a given JUnit test; provide descriptive information that consists of the main motive of test
\item the property to be tested in the test, and the prerequisite needed in the test or the object to be tested (see~\cref{sec:test-pattern-section} for details) to facilitate the improvement of non-descriptive test names
\end{enumerate*}.
%
Unlike existing approaches that suggesting improvements, which were designed to handle general methods, our approach is specific to JUnit tests.
%
The narrower scope of the work allows it to take advantage of the highly repetitive structures that exist in both test names and bodies of JUnit tests (see~\cref{sec:test-pattern-section}).
%
From a high-level point of view, the pattern-based approach uses a set of predefined patterns to extract descriptive information from both a test's name and body.
%
This information is then compared to find non-descriptive names (i.e., cases where the name does not accurately summarize the body).
%
When a mismatch is found, the information used by the approach can help developers address the mismatch and improve the quality of the test name.


Finally, for the remaining future work, I plan to use a set of top-level and secondary codes to build an IDE plugin-based implementation of the concept-based approach and start to conduct a full-scale evaluation of the approach by using the implementation.
%
Furthermore, I plan to propose an innovative, shape-based approach to investigate if the physical appearance (i.e., length, camel cases, etc.), descriptiveness, uniqueness of test names can affect developer's comprehension of the corresponding tests.