\section{Introduction}
\label{sec:introduction}

Unit tests are an important artifact that supports the software development process in several ways.
% change done here 
In addition to helping developers ensure the quality of their software by checking for failures~\cite{daka2014survey}, they can also serve as an important source of documentation not only for human developers but also for automated software engineering tools (e.g., recent work on fault localization by \citeauthor{li2019deepfl} uses test name information~\cite{li2019deepfl}).
%
For example, when a test fails, its name can provide the first step towards understanding the purpose of the test and ultimately fixing the cause of the observed failure.
%
Similarly, a test's name can help developers decide whether a test should be left alone, modified, or removed in response to changes in the application under test and whether the test should be included in a regression test suite.


In this proposal, we believe that test names are \enquote{good} if they are descriptive (e.g, they accurately summarize both the scenario and the expected outcome of the test~\cite{trenk14}) and \enquote{bad} if they are not descriptive.
%
This is because descriptive names:
\begin{enumerate*}
\item make it easier to tell if some functionality is not being tested---if a behavior is not mentioned in the name of a test, then the behavior is not being tested
\item help prevent tests that are too large or contain unrelated assertions---if a test cannot be summarized, it likely should be split into multiple tests
\item serve as documentation for the class under test---a class's supported functionality can be identified by reading the names of its tests
\end{enumerate*}~\cite{zhang2015automatically}.


Unfortunately, unit tests often lack descriptive names.
%
For example, an exploratory study by \citeauthor{zhang2015automatically} found that only \SI{9}{\percent} of the \num{213423} test names they considered were complete (i.e., fully described the body of test) while \SI{62}{\percent} were missing some information and \SI{29}{\percent} contained no useful information (e.g., tests named \enquote{test})~\cite{zhang2015automatically}.
%
Poor test names can be due to developers writing non-descriptive or incomplete names.
%
They can also occur due to incomplete code modifications.
%
For example, a developer may modify a test's body but fail to make the corresponding changes to the test's name.
%
Regardless of the cause, non-descriptive test names complicate comprehension tasks and increase the costs and difficulty of software development.


Because non-descriptive names negatively impact software development, there have been several attempts to address this issue.
%
One approach has been to automatically generate names based on implementations (e.g.,~\cite{arcuri2014automated, zhang2015automatically, daka2017generating}).
%
For example, \citeauthor{zhang2015automatically} and \citeauthor{daka2017generating} use static and dynamic analysis, respectively, to extract important expressions from a test's body and natural language processing techniques to transform such expression into test names \cite{zhang2015automatically, daka2017generating}. 
%
Another approach is to help developers improve their existing names by suggesting improvements.
%
For example, \citeauthor{host2009debugging} proposed an approach for Java methods and variables which uses a set of naming rules and related semantics~\cite{host2009debugging}, \citeauthor{li2019deepfl} provided a learning-based approach to locate software faults using test name information \cite{li2019deepfl}, and \citeauthor{allamanis2015suggesting} and \citeauthor{pradel2018deepbugs} use a model-based and a learning-based approach, respectively, to directly suggest better names or find name-based bugs to facilitate improvements~\cite{allamanis2015suggesting, pradel2018deepbugs}.


As the first part of this proposal, I propose a new, pattern-based approach that can:
\begin{enumerate*}
\item detect non-descriptive test names by finding information mismatches between the test name and body of a given JUnit test and provide descriptive information that is a summarization of the action, predicate, and scenario of the test body
\item use the descriptive information to facilitate the improvement of non-descriptive test names
\end{enumerate*}~\cite{wu2020pattern}.
%
Unlike existing approaches that suggesting improvements, which were designed to handle general methods, our approach is specific to JUnit tests.
%
The narrower scope of the work allows it to take advantage of the highly repetitive structures that exist in both test names and bodies of JUnit tests (see~\cref{sec:test-pattern-section}).
%
From a high-level point of view, the pattern-based approach uses a set of predefined patterns to extract descriptive information from both a test's name and body.
%
This information is then compared to find non-descriptive names (i.e., cases where the name does not accurately summarize the body).
%
When a mismatch is found, the information used by the approach can help developers address the mismatch and improve the quality of the tests by constructing descriptive names that can summarize the descriptive information from the test body.


While automatically generating names from bodies eliminates the possibility of mismatches between names and bodies the generated names do not always meet with developer approval (e.g., they may not fit with existing naming conventions).
%
The test name\slash body summarization approaches (e.g., our pattern-based approach) and most of the existing approaches~\cite{arcuri2014automated,zhang2015automatically,allamanis2015suggesting,daka2017generating,li2019deepfl} can provide suggestions for improving non-descriptive names by a set of patterns or generate descriptive names by a series of pre-defined rules and models.
%
However, none of those work considered the existing naming conventions of test names (i.e., the naming rationales) that are commonly used by developers when constructing test names.
%
Therefore, I further investigate how to automatically generate descriptive test names that follows the most common naming rationale used by developers.
%
Judging from the collected results of our empirical studies~\cref{sec:empStudy}, different developers often use different naming rationales for constructing test names, but most of the existing test names are named after what makes the test unique, which are using the unique information extracted from the test body (i.e., the uniqueness of the test).
%
So as to generate descriptive names for unit tests (i.e., a descriptive test name can summarize important parts of the test body that are unique in its test class), I propose a concept-based approaches that can be used to provide descriptive names with the uniqueness of test.


For the second part of this proposal, I propose the following research tasks, which will contribute to a concept-based approaches that can provide a uniqueness-based naming rationale for unit tests.
%
First, in order to show the evidence that different developers often use different naming rationales to name their tests, an empirical study was conducted as the first step of building the approach, and a set of selective codes were also produced as part of the results of the empirical study.
%
We found out that most of the inspected tests are indeed named after what makes them unique, and the uniqueness of tests can be identified and utilized for future improvement by a set of top-level and secondary codes, which came from the selective coding process using basic Java code elements as shown in~\cref{sec:test-pattern-section}.
%
Therefore, in order to generate descriptive test names using the uniqueness of test, I propose a novel, concept-based approach that can:
%
\begin{enumerate*}
\item verify whether a test is named after what makes it unique or not
\item utilize the uniqueness of tests from the tagged text of tests to provide a uniqueness-based naming rationale for developers
\item generate straightforward results using formal concept analysis (FCA)~\cite{ganter2012formal} to facilitate improvement of existing test names (i.e., provide descriptive names)
\end{enumerate*}~\cite{emp-study}.


The following steps to complete the concept-based approach will be building an implementation of the approach and performing an empirical evaluation of the implemented approach with quantitative analysis.
%
So for the rest of planned future work, I plan to use a set of top-level and secondary codes to build an IDE plugin implementation of the concept-based approach and conduct the empirical evaluation of the approach by using the implementation of it.
%
Our approach is not only specific to JUnit tests but also one of the pioneers to focus on providing a uniqueness-based naming rationale to help developers improve existing test names.
%
From a high-level point of view, the concept-based approach uses a set of selective codes for tagging unit tests, which can also extract unique information from the test.
%
And that unique information can be used to generate descriptive test names that are based on the uniqueness of the test bodies.
%
In the process of using the selective codes, the approach utilizes formal concept analysis (FCA) to mimic the formal abstractions of concepts of human thoughts, which would help us to generate descriptive names in the same fashion as a human developer.
%
FCA will take the tagged text produced by the selective codes as input for building the formal concepts to generate descriptive test names (i.e., test names that contains unique information).
%
The unique information can be used to either improve existing test names that are lack of the unique information or generate descriptive test names with unique information (i.e., summarize all the information from the tagged text to construct a specific name).


Finally, I plan to propose an innovative, class under test (CUT)-based approach to investigate if we can use a similar set of selective codes to extract the descriptive information from the associated classes under test of each test for deciding if the test is actually named after something that makes it unique but is out of the scope of its test body.
%
The CUT-based approach will follow these steps:
\begin{enumerate*}
\item trace every class under test that is related to the test using the variables and method calls in its test body
\item utilize a set of selective codes to parse the variables and methods in those classes under test and extract descriptive information
\item check if the test is named after that information; if it is, provide the information for developers; if not, construct a descriptive name using the information for the test
\item it is also possible to use this approach to provide further additional information for constructing descriptive names
\end{enumerate*}.
%
A different empirical evaluation will also be conducted to verify the validity of the CUT-based approach.