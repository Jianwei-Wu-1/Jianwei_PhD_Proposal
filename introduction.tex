\section{Introduction}
\label{sec:introduction}

Unit testing is one of the primary methods developers use to measure the correctness of software by writing a dedicated test for each individual component of the software. 
%
As the vast adoption of unit testing in the process of software development, ensuring the quality and readability of unit tests has become one of the most crucial goal in software testing research.
%
Because many unit tests are still manually written for many existing projects, understanding them still remains as a tedious and error-prone activity in the software development process.
%
Moreover, several reports claimed that the costs of software testing can account for over \num{50}\% in the total cost of developing software~\cite{anand2013orchestrated}, and the majority of that cost is due to poor quality and readability.
%
Regardless of the decades of work that was devoted to developing automated or rule-based approaches that attempt to generate better test names for developers to use, the descriptiveness and uniqueness of test names are still low.
%
Therefore, additional approaches that can provide descriptive and unique names for unit tests would be very useful for not only improving the quality and readability of unit tests but also reducing the cost of performing unit testing in the software development process.


Thankfully, many approaches has been provided by the software testing research community, which are designed to help developers write better names (e.g., \cite{host2009debugging,allamanis2014learning,pradel2018deepbugs,zhang2016towards,fraser2011evosuite,thummalapenta2009mseqgen,daka2017generating,allamanis2015suggesting}).
%
Although these novel approaches are very good at helping developers construct better test names at some level, their generated names are still lacking of descriptiveness and uniqueness.
%
Therefore, in order to mitigate the poor quality and readability of test names, it is crucial to build new approaches that can both help developers write better test names and improve existing test names, which are currently not descriptive and unique.


In this proposal, I believe that test names in unit tests are \enquote{good} if they are descriptive (i.e., they accurately summarize both the scenario and the expected outcome of the test~\cite{trenk14}) and unique (i.e., being unique among its sibling tests).
%
First, the descriptive names are \enquote{good} because they can:
\begin{enumerate*}
\item make it easier to tell if some functionality is not being tested---if a behavior is not mentioned in the name of a test, then the behavior is not being tested
\item help prevent tests that are too large or contain unrelated assertions---if a test cannot be summarized, it likely should be split into multiple tests
\item serve as documentation for the class under test---a class's supported functionality can be identified by reading the names of its tests
\end{enumerate*}~\cite{zhang2015automatically}.
%
Second, rather than limiting to the descriptiveness of test names, an empirical study was conducted to determine whether tests are named after what makes them unique.
%
From the resulting data of the empirical study, the unique names are \enquote{good} because they can:
\begin{enumerate*}
\item can be used to differentiate each test from its sibling tests
\item help developers to understand the intention of creating a test
\item help to detect duplicated tests
\end{enumerate*}.
%
Last, another empirical study will be conducted to further extend the idea of providing descriptive and unique names to other components of unit tests (e.g., test class names and identifier names).
%
In my dissertation work, I will focus on systematically providing descriptive and unique names for unit tests.
%
More specifically, I propose the following research tasks:


First, I propose a pattern-based approach to improve non-descriptive test names.
%
The test patterns in the approach are mined from a large corpus of unit tests (i.e., both names and bodies), and I subtracted the commonalities in the test names and bodies as test patterns.
%
By applying the test patterns to the unit tests, the pattern-based approach can extract the descriptive information from the tests and compare the extracted information between the name and body to determine if the name is descriptive.


Second, I propose a concept-based approach to extract the uniqueness of tests to construct unique names or improve test names that were not previously named after the uniqueness of tests.
%
An empirical study is first conducted to answer two research questions:
\begin{enumerate*}
\item what makes a test unique
\item whether tests are named after what makes them unique
\end{enumerate*}.
%
From the results of the empirical study, most of the inspected test names are indeed named after what makes them unique, and the uniqueness of tests are identified and refined with a set of high-level and secondary codes.


Finally, in the remaining work, I plan to use the set of high-level and secondary codes to build an IDE plugin-based implementation of the concept-based approach and start to conduct a full-scale evaluation of the approach by using the implementation.
%
Moreover, I plan to propose a new approach to provide descriptive and unique names that is not limited to test names but for other components, such as test class names and variable names, in unit tests.
%
By using another empirical study, I will not only explore the approach to provide descriptive and unique names for other components in unit tests but also further investigate the relationship between test names and other components' names.


